\documentclass{beamer}
 
\usepackage[utf8]{inputenc}
\usepackage{svg}
\usepackage{tikz}
\usepackage{amsmath}
\usepackage[customcolors]{hf-tikz}


\definecolor{amber}{rgb}{1.0, 0.75, 0.0}
\definecolor{generator}{rgb}{0.67, 0.9, 0.93}
\definecolor{discriminator}{rgb}{0.89, 0.44, 0.48}

\usetheme{metropolis}
 
 
\title[GAN] %optional
{GAN - Theory and Applications}
  
\author % (optional, for multiple authors)
{Michele de Simoni  \and  \\ Paolo Galeone \and \\ Emanuele Ghelfi \and \\ Federico di Mattia}
 

 
\titlegraphic{
	 \begin{picture}(0,0)
	\put(100,25){\makebox(0,0)[rt]{
	\includegraphics[height=1.5cm]{zuru-logo.png}}}
\end{picture}
}
 
 
 
\begin{document}
 
{
  \usebackgroundtemplate{
  	\tikz[overlay,remember picture] \node[opacity=0.2, at=(current page.center)] {
  		\includesvg[width=0.5\textwidth]{pycon_x_logo.svg}};}
  \begin{frame}
    \titlepage
  \end{frame}
}

 
\begin{frame}
\frametitle{Generative Adversarial Networks}

\setbeamercolor{block body}{bg=amber!20!white}
\begin{block}{}
	{\large ``Adversarial Training (also called GAN for Generative Adversarial Networks) is the most interesting idea in the last 10 years of ML.''}
	\vskip5mm
	\hspace*\fill{\small--- Yann LeCun}
\end{block}

\end{frame}

\begin{frame}
\frametitle{Generative Adversarial Networks}
	Two components, the \textbf{generator} and the \textbf{discriminator}:
	\begin{itemize}
		\item The \textbf{generator} G, aim is to capture the data distribution.
		\item The \textbf{discriminator} D, estimates the probability that a sample came from the training data rather than from G.
	\end{itemize}

\begin{figure}
	\includegraphics[width=\textwidth]{GANs.png}
	\caption{Credits: Reference  }
\end{figure}

\end{frame}

\begin{frame}
\frametitle{Generative Adversarial Networks}
Generator and Discriminator compete against each other, playing the following zero sum min-max game with value function $V_{GAN}(D,G)$:
\begin{equation}
 \min_G \max_D V_{GAN}(D,G) =\underset{x \sim p_{data}(x)}{\mathbb{E}} [\log D(x)]  + \underset{z \sim p_z(z)}{\mathbb{E}}[\log(1 - D(G(z)))] 
\end{equation}

\end{frame}

\begin{frame}
\frametitle{Generative Adversarial Networks}
Intuitive explanation:
\begin{itemize}
	\item Discriminator needs to:
	\begin{itemize}
		\hfsetfillcolor{discriminator!10}
		\hfsetbordercolor{discriminator}
		\item Correctly classify real data: \\ \textbf{maximize} \begin{equation} \tikzmarkin{a}(0.2,-0.5)(-0.2,0.65) \underset{x \sim p_{data}(x)}{\mathbb{E}} [\log D(x)].\tikzmarkend{a}
		\end{equation}.
		\item Correctly classify wrong data: \\ \textbf{maximize} \begin{equation} \tikzmarkin{b}(0.2,-0.5)(-0.2,0.65)\underset{z \sim p_z(z)}{\mathbb{E}}[\log(1 - D(G(z)))].\tikzmarkend{b} \end{equation}
	\end{itemize}
\end{itemize}
\end{frame}
 
\end{document}